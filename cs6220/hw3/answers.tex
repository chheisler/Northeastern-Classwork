\documentclass[12pt]{article}
\title{Homework 3}
\author{Charles Heisler -- CS6220 -- 11/7/2016}
\date{}
\usepackage[margin=1in]{geometry}
\usepackage{amsmath}
\usepackage{xfrac}
\usepackage{enumitem}
\usepackage{graphicx}

\begin{document}
	\maketitle

	\section*{1.}
	\subsection*{(a)}
	\begingroup
	\addtolength{\jot}{1em}
	\begin{align*}
		m_1 &= \frac{1}{\lvert C_1 \rvert}\sum_{x_i \in C_1}x_i \\
		&= \frac{\left(5,5\right)+\left(8,7\right)+\left(7,3\right)}{3} \\
		&= \mathbf{\left(6\sfrac{2}{3},5\right)}
	\end{align*}
	\endgroup

	\begingroup
	\addtolength{\jot}{1em}
	\begin{align*}
		m_2 &= \frac{1}{\lvert C_2 \rvert}\sum_{x_i \in C_2}x_i \\
		&= \frac{\left(6,5\right)+\left(4,4\right)+\left(9,2\right)+\left(3,5\right)+\left(8,4\right)}{5} \\
		&= \mathbf{\left(6,4\right)}
	\end{align*}
	\endgroup

	\subsection*{(b)}
	\begingroup
	\addtolength{\jot}{1em}
	\begin{align*}
		m &= \frac{1}{\lvert C_1 \cup C_2 \rvert}\sum_{x_i \in C_1 \cup C_2}x_i \\
		&= \frac{\left(5,5\right)+\left(8,7\right)+\left(7,3\right)+\left(6,5\right)+\left(4,4\right)+\left(9,2\right)+\left(3,5\right)+\left(8,4\right)}{8} \\
		&= \mathbf{\left(6\sfrac{1}{4},4\sfrac{3}{8}\right)}
	\end{align*}
	\endgroup

	\subsection*{(c)}
	\begingroup
	\addtolength{\jot}{1em}
	\begin{align*}
		S_1 &= \sum_{x_i \in C_1} \left(c_i-m_1\right)\left(x_i-m_1\right)^\textrm{T} \\
		&= \begin{bmatrix}
			2\sfrac{7}{9} & 0 \\
			0 & 0
		\end{bmatrix}
		+ \begin{bmatrix}
			1.\sfrac{7}{9} & 2\sfrac{2}{3} \\
			2.\sfrac{2}{3} & 4
		\end{bmatrix}
		+ \begin{bmatrix}
			sfrac{1}{9} & -\sfrac{2}{3} \\
			-\sfrac{2}{3} & 4
		\end{bmatrix} \\
		&= \begin{bmatrix}
			\mathbf{4.\sfrac{2}{3}} & \mathbf{2} \\
			\mathbf{2} & \mathbf{8}
		\end{bmatrix}
	\end{align*}
	\endgroup

	\begingroup
	\addtolength{\jot}{1em}
	\begin{align*}
		S_2 &= \sum_{x_i \in C_2}\left(x_i-m_2\right)\left(x_i-m_2\right)^\textrm{T} \\
		&= \begin{bmatrix}
			0 & 0 \\
			0 & 1
		\end{bmatrix}
		+ \begin{bmatrix}
			4 & 0 \\
			0 & 0
		\end{bmatrix}
		+ \begin{bmatrix}
			9 & -6 \\
			-6 & 4
		\end{bmatrix}
		+ \begin{bmatrix}
			9 & -3 \\
			-3 & 1
		\end{bmatrix}
		+ \begin{bmatrix}
			4 & 0 \\
			0 & 0
		\end{bmatrix} \\
		& = \begin{bmatrix}
			\mathbf{26} & \mathbf{-9} \\
			\mathbf{-9} & \mathbf{6}
		\end{bmatrix}
	\end{align*}
	\endgroup

	\subsection*{(d)}
	\begingroup
	\addtolength{\jot}{1em}
	\begin{align*}
		S_W &= S_1 + S_2 \\
		&= \begin{bmatrix}
			4\sfrac{2}{3} & 2 \\
			2 & 8
		\end{bmatrix}
		+ \begin{bmatrix}
			26 & -9 \\
			-9 & 6
		\end{bmatrix} \\
		&= \begin{bmatrix}
			\mathbf{30\sfrac{2}{3}} & \mathbf{-7} \\
			\mathbf{-7} & \mathbf{14}
		\end{bmatrix}
	\end{align*}

	\subsection*{(e)}
	\begingroup
	\addtolength{\jot}{1em}
	\begin{align*}
		S_B &= \lvert C_1 \rvert \left(m_1-m\right)\left(m_1-m\right)^\textrm{T}
		+ \lvert C_2 \rvert \left(m_2-m\right)\left(m_2-m\right)^\textrm{T} \\
		&= 3 \begin{bmatrix}
			\sfrac{25}{144} & \sfrac{25}{96} \\
			\sfrac{25}{96} & \sfrac{25}{64}
		\end{bmatrix}
		+ 5 \begin{bmatrix}
			\sfrac{1}{16} & \sfrac{15}{32} \\
			\sfrac{15}{32} & \sfrac{45}{64}
		\end{bmatrix} \\
		&= \begin{bmatrix}
			\mathbf{\sfrac{5}{6}} & \mathbf{1\sfrac{1}{4}} \\
			\mathbf{1\sfrac{1}{4}} & \mathbf{1\sfrac{7}{8}}
		\end{bmatrix}
 	\end{align*}
	\endgroup

	\subsection*{(f)}
	\begingroup
	\addtolength{\jot}{1em}
	\begin{align*}
		\frac{\textrm{tr}\left(S_B\right)}{\textrm{tr}\left(S_W\right)}
		= \frac{\sfrac{5}{6}+1\sfrac{7}{8}}{30\sfrac{2}{3}+14}
		= \mathbf{\sfrac{65}{1072}}
	\end{align*}
	\endgroup

	\section*{2.}
	\subsection*{(a)}
	There is one cluster ${(0,1),(5,2),(2,3),(6,1),(3,4),(6,3),(7,2),(1,2)}$.

	\subsection*{(b)}
	Every pair of points within the cluster is density-connected.

	\subsection*{(c)}
	The points $(10,2)$, $(0,6)$ and $(0.7)$ are noise.

	\section*{3.}
	Let, $x_1 = \textrm{HHTT}$ $x_2 = \textrm{HHHH}$ $x_3 = \textrm{TTTT}$
	
	\subsection*{Iteration 1.}

	\subsubsection*{E-Step.}

	\begin{align*}
		\gamma_{1,A} &\propto p\left(x_1 \vert A\right) p\left(A\right)
		= \mu_A^2 \left(1-\mu_A \right)^2 \pi
		= 0.55^2 \cdot \left(1-0.55\right)^2 \cdot 0.6
		\approx 0.036754 \\
		\gamma_{2,A} &\propto p\left(x_2 \vert A\right) p\left(A\right)
		= \mu_A^4 \left(1-\mu_A\right)^0 \pi
		= 0.55^4 \cdot \left(1-0.55\right)^0 \cdot 0.6
		\approx 0.054904 \\
		\gamma_{3,A} &\propto p\left(x_3 \vert A\right) p\left(A\right)
		= \mu_A^0 \left(1-\mu_A\right)^4 \pi
		= 0.55^0 \cdot \left(1-0.55\right)^4 \cdot 0.6
		\approx 0.024604
	\end{align*}

	\begin{align*}
		\gamma_{1,B} &\propto p\left(x_1 \vert B\right) p\left(B\right)
		= \mu_B^2 \left(1-\mu_B \right)^2 \left(1-\pi\right)
		= 0.45^2 \cdot \left(1-0.45\right)^2 \cdot \left(1-0.6\right)
		\approx 0.024503 \\
		\gamma_{2,B} &\propto p\left(x_2 \vert B\right) p\left(B\right)
		= \mu_B^4 \left(1-\mu_B\right)^0 \left(1-\pi\right)
		= 0.45^4 \cdot \left(1-0.45\right)^0 \cdot \left(1-0.6\right)
		\approx 0.016403 \\
		\gamma_{3,B} &\propto p\left(x_3 \vert B\right) p\left(B\right)
		= \mu_B^0 \left(1-\mu_B\right)^4 \left(1-\pi\right)
		= 0.45^0 \cdot \left(1-0.45\right)^4 \cdot \left(1-0.6\right)
		\approx 0.036603
	\end{align*}

	\begin{align*}
		\gamma_{1,A} &= \frac{p\left(x_1 \vert A\right) p\left(A\right)}
		{p\left(x_1 \vert A\right) p\left(A\right)
		+ p\left(x_1 \vert B\right) p\left(B\right)}
		\approx \frac{0.036754}{0.036754+0.024503} = 0.6 \\
		\gamma_{2,A} &= \frac{p\left(x_2 \vert A\right) p\left(A\right)}
		{p\left(x_2 \vert A\right) p\left(A\right)
		+ p\left(x_2 \vert B\right) p\left(B\right)}
		\approx \frac{0.054904}{0.054904+0.016403} \approx 0.769971 \\
		\gamma_{3,A} &= \frac{p\left(x_3 \vert A\right) p\left(A\right)}
		{p\left(x_3 \vert A\right) p\left(A\right)
		+ p\left(x_3 \vert B\right) p\left(B\right)}
		\approx \frac{0.024604}{0.024604+0.036603} \approx 0.401981
	\end{align*}

	\begin{align*}
		\gamma_{1,B} &= \frac{p\left(x_1 \vert B\right) p\left(B\right)}
		{p\left(x_1 \vert A\right) p\left(A\right)
		+ p\left(x_1 \vert B\right) p\left(B\right)}
		\approx \frac{0.024503}{0.036754+0.024503} = 0.4 \\
		\gamma_{2,B} &= \frac{p\left(x_2 \vert B\right) p\left(B\right)}
		{p\left(x_2 \vert A\right) p\left(A\right)
		+ p\left(x_2 \vert B\right) p\left(B\right)}
		\approx \frac{0.016403}{0.054904+0.016403} \approx 0.230029 \\
		\gamma_{3,B} &= \frac{p\left(x_3 \vert B\right) p\left(B\right)}
		{p\left(x_3 \vert A\right) p\left(A\right)
		+ p\left(x_3 \vert B\right) p\left(B\right)}
		\approx \frac{0.036603}{0.024604+0.036603} \approx 0.598019
	\end{align*}

	\subsubsection*{M-Step.}
	
	\begin{align*}
		\mu_A &= \frac{0.5\gamma_{1,A}+1\gamma_{2,A}+0\gamma_{3,A}}{\gamma_{1,A}+\gamma_{2,A}+\gamma_{3,A}}
		\approx \frac{0.5 \cdot 0.6+1\cdot 0.769971+0\cdot 0.401981}{0.6+0.769971+0.401981} \approx 0.603837 \\
		\mu_B &= \frac{0.5\gamma_{1,B}+1\gamma_{2,B}+0\gamma_{3,B}}{\gamma_{1,B}+\gamma_{2,B}+\gamma_{3,B}}
		\approx \frac{0.5 \cdot 0.4 + 1 \cdot 0.230029 + 0 \cdot 0.598019}{0.4 + 0.230029 + 0.598019} \approx 0.350173 \\
		\pi &= \frac{\gamma_{1,A}+\gamma_{2,A}+\gamma_{3,A}}{3} \approx \frac{0.6+0.769971+0.401981}{3} \approx 0.590651
	\end{align*}

	\subsection*{Iteration 2.}

	\subsubsection*{E-Step.}

	\begin{align*}
		\gamma_{1,A} &\propto p\left(x_1 \vert A\right) p\left(A\right)
		= \mu_A^2 \left(1-\mu_A \right)^2 \pi
		\approx 0.603837^2 \cdot \left(1-0.603837\right)^2 \cdot 0.590651
		\approx 0.033800 \\
		\gamma_{2,A} &\propto p\left(x_2 \vert A\right) p\left(A\right)
		= \mu_A^4 \left(1-\mu_A\right)^0 \pi
		\approx 0.603837^4 \cdot \left(1-0.603837\right)^0 \cdot 0.590651
		\approx 0.078525 \\
		\gamma_{3,A} &\propto p\left(x_3 \vert A\right) p\left(A\right)
		= \mu_A^0 \left(1-\mu_A\right)^4 \pi
		\approx 0.603837^0 \cdot \left(1-0.603837\right)^4 \cdot 0.590651
		\approx 0.14549
	\end{align*}

	\begin{align*}
		\gamma_{1,B} &\propto p\left(x_1 \vert B\right) p\left(B\right)
		= \mu_B^2 \left(1-\mu_B \right)^2 \left(1-\pi\right)
		\approx 0.350173^2 \cdot \left(1-0.350173\right)^2 \cdot \left(1-0.590651\right)
		\approx 0.021196 \\
		\gamma_{2,B} &\propto p\left(x_2 \vert B\right) p\left(B\right)
		= \mu_B^4 \left(1-\mu_B\right)^0 \left(1-\pi\right)
		\approx 0.350173^4 \cdot \left(1-0.350173\right)^0 \cdot \left(1-0.590651\right)
		\approx 0.006155 \\
		\gamma_{3,B} &\propto p\left(x_3 \vert B\right) p\left(B\right)
		= \mu_B^0 \left(1-\mu_B\right)^4 \left(1-\pi\right)
		\approx 0.350173^0 \cdot \left(1-0.350173\right)^4 \cdot \left(1-0.590651\right)
		\approx 0.072994
	\end{align*}

	\begin{align*}
		\gamma_{1,A} &= \frac{p\left(x_1 \vert A\right) p\left(A\right)}
		{p\left(x_1 \vert A\right) p\left(A\right)
		+ p\left(x_1 \vert B\right) p\left(B\right)}
		\approx \frac{0.033800}{0.033800+0.021196} \approx 0.614591 \\
		\gamma_{2,A} &= \frac{p\left(x_2 \vert A\right) p\left(A\right)}
		{p\left(x_2 \vert A\right) p\left(A\right)
		+ p\left(x_2 \vert B\right) p\left(B\right)}
		\approx \frac{0.078525}{0.078525+0.006155} \approx 0.927315 \\
		\gamma_{3,A} &= \frac{p\left(x_3 \vert A\right) p\left(A\right)}
		{p\left(x_3 \vert A\right) p\left(A\right)
		+ p\left(x_3 \vert B\right) p\left(B\right)}
		\approx \frac{0.14549}{0.14549+0.072993} \approx 0.166191
	\end{align*}

	\begin{align*}
		\gamma_{1,B} &= \frac{p\left(x_1 \vert B\right) p\left(B\right)}
		{p\left(x_1 \vert A\right) p\left(A\right)
		+ p\left(x_1 \vert B\right) p\left(B\right)}
		\approx \frac{0.021196}{0.033800+0.021196} \approx 0.385409 \\
		\gamma_{2,B} &= \frac{p\left(x_2 \vert B\right) p\left(A\right)}
		{p\left(x_2 \vert A\right) p\left(A\right)
		+ p\left(x_2 \vert B\right) p\left(B\right)}
		\approx \frac{0.006155}{0.078525+0.006155} \approx 0.072685 \\
		\gamma_{3,B} &= \frac{p\left(x_3 \vert B\right) p\left(A\right)}
		{p\left(x_3 \vert A\right) p\left(A\right)
		+ p\left(x_3 \vert B\right) p\left(B\right)}
		\approx \frac{0.072993}{0.14549+0.072993} \approx 0.833809
	\end{align*}

	\subsubsection*{M-Step.}

	\begin{align*}
		\mu_A &= \frac{0.5\gamma_{1,A}+1\gamma_{2,A}+0\gamma_{3,A}}{\gamma_{1,A}+\gamma_{2,A}+\gamma_{3,A}}
		\approx \frac{0.5 \cdot 0.614591+1\cdot 0.927315+0\cdot 0.166191}{0.614591+0.927315+0.166191} \approx 0.722799 \\
		\mu_B &= \frac{0.5\gamma_{1,B}+1\gamma_{2,B}+0\gamma_{3,B}}{\gamma_{1,B}+\gamma_{2,B}+\gamma_{3,B}}
		\approx \frac{0.5 \cdot 0.385409 + 1 \cdot 0.072685 + 0 \cdot 0.833809}{0.385409 + 0.072685 + 0.833809} \approx 0.205425 \\
		\pi &= \frac{\gamma_{1,A}+\gamma_{2,A}+\gamma_{3,A}}{3}\approx\frac{0.614591+0.927315+0.166191}{3}\approx 0.569366
	\end{align*}

	\subsection*{Iteration 3.}

	\subsubsection*{E-Step.}

	\begin{align*}
		\gamma_{1,A} &\propto p\left(x_1 \vert A\right) p\left(A\right)
		= \mu_A^2 \left(1-\mu_A \right)^2 \pi
		\approx 0.722799^2 \cdot \left(1-0.722799\right)^2 \cdot 0.569366
		\approx 0.022857 \\
		\gamma_{2,A} &\propto p\left(x_2 \vert A\right) p\left(A\right)
		= \mu_A^4 \left(1-\mu_A\right)^0 \pi
		= 0.722799^4 \cdot \left(1-0.722799\right)^0 \cdot 0.569366
		\approx 0.155404 \\
		\gamma_{3,A} &\propto p\left(x_3 \vert A\right) p\left(A\right)
		= \mu_A^0 \left(1-\mu_A\right)^4 \pi
		\approx 0.722799^0 \cdot \left(1-0.722799\right)^4 \cdot 0.569366
		\approx 0.003361
	\end{align*}

	\begin{align*}
		\gamma_{1,B} &\propto p\left(x_1 \vert B\right) p\left(B\right)
		= \mu_B^2 \left(1-\mu_B \right)^2 \left(1-\pi\right)
		\approx 0.205425^2 \cdot \left(1-0.205425\right)^2 \cdot \left(1-0.569366\right)
		\approx 0.011473 \\
		\gamma_{2,B} &\propto p\left(x_2 \vert B\right) p\left(B\right)
		= \mu_B^4 \left(1-\mu_B\right)^0 \left(1-\pi\right)
		\approx 0.205425^4 \cdot \left(1-0.205425\right)^0 \cdot \left(1-0.569366\right)
		\approx 0.000767 \\
		\gamma_{3,B} &\propto p\left(x_3 \vert B\right) p\left(B\right)
		= \mu_B^0 \left(1-\mu_B\right)^4 \left(1-\pi\right)
		\approx 0.205425^0 \cdot \left(1-0.205425\right)^4 \cdot \left(1-0.569366\right)
		\approx 0.171652
	\end{align*}

	\begin{align*}
		\gamma_{1,A} &= \frac{p\left(x_1 \vert A\right) p\left(A\right)}
		{p\left(x_1 \vert A\right) p\left(A\right)
		+ p\left(x_1 \vert B\right) p\left(B\right)}
		\approx \frac{0.022857}{0.022857+0.011473} \approx 0.665797 \\
		\gamma_{2,A} &= \frac{p\left(x_2 \vert A\right) p\left(A\right)}
		{p\left(x_2 \vert A\right) p\left(A\right)
		+ p\left(x_2 \vert B\right) p\left(B\right)}
		\approx \frac{0.155404}{0.155404+0.000767} \approx 0.995090 \\
		\gamma_{3,A} &= \frac{p\left(x_3 \vert A\right) p\left(A\right)}
		{p\left(x_3 \vert A\right) p\left(A\right)
		+ p\left(x_3 \vert B\right) p\left(B\right)}
		\approx \frac{0.003361}{0.003361+0.171652} \approx 0.019209
	\end{align*}

	\begin{align*}
		\gamma_{1,B} &= \frac{p\left(x_1 \vert B\right) p\left(B\right)}
		{p\left(x_1 \vert A\right) p\left(A\right)
		+ p\left(x_1 \vert B\right) p\left(B\right)}
		\approx \frac{0.011473}{0.022857+0.011473} \approx 0.334203 \\
		\gamma_{2,B} &= \frac{p\left(x_2 \vert B\right) p\left(B\right)}
		{p\left(x_2 \vert A\right) p\left(A\right)
		+ p\left(x_2 \vert B\right) p\left(B\right)}
		\approx \frac{0.000767}{0.155404+0.000767} \approx 0.004910 \\
		\gamma_{3,B} &= \frac{p\left(x_3 \vert B\right) p\left(B\right)}
		{p\left(x_3 \vert A\right) p\left(A\right)
		+ p\left(x_3 \vert B\right) p\left(B\right)}
		\approx \frac{0.171652}{0.003361+0.171652} \approx 0.980791
	\end{align*}

	\subsubsection*{M-Step.}

	\begin{align*}
		\mu_A &= \frac{0.5\gamma_{1,A}+1\gamma_{2,A}+0\gamma_{3,A}}{\gamma_{1,A}+\gamma_{2,A}+\gamma_{3,A}}
		\approx \frac{0.5 \cdot 0.665797+1\cdot 0.995090+0\cdot 0.019209}{0.665797+0.995090+0.019209} \approx 0.790424 \\
		\mu_B &= \frac{0.5\gamma_{1,B}+1\gamma_{2,B}+0\gamma_{3,B}}{\gamma_{1,B}+\gamma_{2,B}+\gamma_{3,B}}
		\approx \frac{0.5 \cdot 0.334203 + 1 \cdot 0.004910 + 0 \cdot 0.980791}{0.334203 + 0.004910 + 0.980791} \approx 0.130321 \\
		\pi &= \frac{\gamma_{1,A}+\gamma_{2,A}+\gamma_{3,A}}{3}\approx\frac{0.665797+0.995090+0.019209}{3}\approx0.560032
	\end{align*}

	\section*{4.}

	\subsection*{(a)}
	$\textrm{AGNES} > \textrm{DBSCAN} > \textrm{GMM} > \textrm{k-means}$

	The clusters are non-convex in shape which makes it difficult for GMM and k-means to fit them.
	While both clusters are well defined and separated, they have different densities so it could
	be difficult to tune DBSCAN to capture both without merging them.

	\subsection*{(b)}
	$\textrm{GMM} > \textrm{k-means} > \textrm{DBSCAN} > \textrm{AGNES}$
	
	The clusters are convex and eliptical, so GMM and k-means should work well. The clusters are
	close together with some overlap, so DBSCAN and AGNES may join separate clusters.

	\subsection*{(c)}
	$\textrm{GMM} > \textrm{k-means} > \textrm{DBSCAN} = \textrm{AGNES}$

	The clusters are roughly convex and suitable for finding with GMM and k-means. There is a lot
	of overlap between the clusters and no clear density break so DBSCAN and AGNES would be prone
	to joining separate clusters.

	\subsection*{(d)}
	$\textrm{GMM} > \textrm{k-means} > \textrm{AGNES} > \textrm{DBSCAN}$
	The clusters are roughly convex and can be clustered with GMM or k-means. The clusters are well
	defined and have few outliers which could be problematic for AGNES. Two of the clusters are joined
	by a strand of data points as dense as the clusters themselves, and could be merged by DBSCAN.

	\subsection*{(e)}
	$\textrm{AGNES} > \textrm{GMM} > \textrm{k-means} > \textrm{DBSCAN}$
	One of the clusters slightly wraps around the other, but it is still largely separable from it with
	GMM or k-means. Between the two clusters there is no obvious break in density so DBSCAN might merge
	them or would have to be tuned so that many points were classified as noise.

	\subsection*{(f)}
	$\textrm{DBSCAN} > \textrm{AGNES} > \textrm{GMM} = \textrm{k-means}$
	One of the clusters is entirely surrounded by another, so they cannot be kept separate with GMM or
	k-means. All of the clusters have well defined areas of dense core points that could be captured
	with DBSCAN, albeit it at the cost of considering some of the outlying points noise.

	\subsection*{(g)}
	$\textrm{DBSCAN} > \textrm{AGNES} > \textrm{GMM} = \textrm{k-means}$
	One of the clusters is wrapped around the other two, so covering it with GMM and k-means without
	including elements from the other two is impossible. There are areas of lower density between
	the clusters, so they can potentially kept separate in DBSCAN. AGNES can account for the shape of
	the clusters, but might have problems with the many outliers in the central two.

	\subsection*{(h)}
	$\textrm{DBSCAN} = \textrm{AGNES} > \textrm{GMM} = \textrm{k-means}$
	The clusters are not convex and covering one with an eliptical GMM or k-means clustering would
	mean including erroneous elements from other clusters. The clusters are well separated with
	uniform density so DBSCAN and AGNES are unlikely to incorrectly join clusters.

	\section*{5.}	

	\subsection*{(a)}
	\includegraphics[scale=0.9]{kmeans.png}

	\subsection*{(b)}
	\begin{tabular}{c | c | c | c}
		$k$ & $\mu_k$ & $\mu_k-2\sigma_k$ & $\mu_k+2\sigma_k$ \\
		\hline & & & \\
		1 & 41562.000000 & 41562.000000 & 41562.000000 \\
		2 & 29158.418442 & 25819.225611 & 32497.611273 \\
		3 & 24369.750977 & 22540.051494 & 26199.450461 \\
		4 & 21063.591137 & 17703.303131 & 24423.879144 \\
		5 & 18123.257696 & 14730.652325 & 21515.863067 \\
		6 & 16757.533816 & 13228.550113 & 20286.517518 \\
		7 & 14546.190494 & 11556.771269 & 17535.609720 \\
		8 & 13311.097774 & 12016.239597 & 14605.955951 \\
		9 & 12270.389369 & 10749.060376 & 13791.718362 \\
		10 & 11685.283172 & 10051.332748 & 13319.233595 \\
		11 & 11150.440644 & 9808.773622 & 12492.107665 \\
		12 & 10509.131092 & 8921.560093 & 12096.702090 
	\end{tabular}

	\subsection*{(c)}
	As $k$ increases and approaches $N$, the SSE approaches zero. This means
	that if SSE is used as a criterion for judging clustering, increasing $k$
	will always improve performance up until the point where $k=N$ and SSE becomes
	zero.

	\subsection*{(d)}
	An alternative measure of compactness would be a scatter criterion like the one
	calculated in part (f) of question 1.
 
\end{document}
